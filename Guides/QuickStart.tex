\documentclass[a4paper]{memoir}

\usepackage[colorlinks]{hyperref}

\title{Architettura degli Elaboratori \\ Quick-Start Guide \\ A.A. 2012/2013}
\author{Michele ``Jazzinghen'' Bianchi \\ eMail: \url{jazzinghen@gmail.com} \\%
	Website \url{http://disi.unitn.it/~bianchi/}}

\usepackage{color,calc,graphicx,soul,fourier}
\usepackage[utf8]{inputenc}
\usepackage[T1]{fontenc}
\usepackage[italian]{babel}

\usepackage{graphicx}

% args: image path
% caption
% width
% label
\newcommand{\image}[4]{%
    \begin{figure}[hbt]%
    \centering%
    \includegraphics[width=#3\columnwidth]{#1}%
    \caption{\emph{#2}}%
    \label{#4}%
    \end{figure}%
}

% args: image path
% caption
% label
\newcommand{\imageFullWidth}[3]{\image{#1}{#2}{0.8}{#3}}
\newcommand{\keyword}[1]{{\color{myorange}#1}}


\usepackage{fancyvrb}
\usepackage[svgnames,dvipsnames]{xcolor}
\definecolor{felinesrcbgcolor}{rgb}{1,1,0.85}
\definecolor{felinesrcbgcolor}{rgb}{0.94,0.97,1}
\definecolor{felineframe}{rgb}{0.79,0.88,1}
\definecolor{myorange}{rgb}{1,0.375,0}
\fvset{frame=lines,
  framesep=3mm,
  framerule=3pt,
  fontsize=\small,
  rulecolor=\color{myorange},
  formatcom=\color{DarkGreen},
}
\lstset{language=[ANSI]C, classoffset=0,
  morekeywords={uint32_t}, keywordstyle=\color{myorange}\textbf}

\hypersetup{
	urlcolor=myorange
}

\definecolor{nicered}{rgb}{.647,.129,.149}
\makeatletter
\newlength\dlf@normtxtw
\setlength\dlf@normtxtw{\textwidth}
\def\myhelvetfont{\def\sfdefault{mdput}}
\newsavebox{\feline@chapter}
\newcommand\feline@chapter@marker[1][4cm]{%
	\sbox\feline@chapter{%
	\resizebox{!}{#1}{\fboxsep=1pt%
	\colorbox{nicered}{\color{white}\bfseries\sffamily\thechapter}%
	}}%
	\rotatebox{90}{%
	\resizebox{%
	\heightof{\usebox{\feline@chapter}}+\depthof{\usebox{\feline@chapter}}}%
	{!}{\scshape\so\@chapapp}}\quad%
	\raisebox{\depthof{\usebox{\feline@chapter}}}{\usebox{\feline@chapter}}%
}
\newcommand\feline@chm[1][4cm]{%
\sbox\feline@chapter{\feline@chapter@marker[#1]}%
\makebox[0pt][l]{% aka \rlap
\makebox[1cm][r]{\usebox\feline@chapter}%
}}
\makechapterstyle{daleif1}{
	\renewcommand\chapnamefont{\normalfont\Large\scshape\raggedleft\so}
	\renewcommand\chaptitlefont{\normalfont\huge\bfseries\scshape\color{nicered}}
	\renewcommand\chapternamenum{}
	\renewcommand\printchaptername{}
	\renewcommand\printchapternum{\null\hfill\feline@chm[2.5cm]\par}
	\renewcommand\afterchapternum{\par\vskip\midchapskip}
	\renewcommand\printchaptertitle[1]{\chaptitlefont\raggedleft ##1\par}
}
\makeatother
\chapterstyle{daleif1}

\begin{document}

\maketitle

\chapter*{Prima d'iniziare}
	
	
	
	\section*{Informazioni sul corso}
	
		Partiamo dalla cosa più importante: come passare l'esame.
		
		Il corso prevede ``solo'' una prova scritta che conterrà un po' tutto quello
		che vedrete (avete visto) nel corso, dalla teoria agli esercizi. Per avere
		una lista di quello che potrebbe esserci controllate sul sito di Luigi
		\footnote{\url{http://disi.unitn.it/~palopoli/courses/Arch/syllabus.html}}
		anche se, come regola di base, direi che sapere le cose che sono presenti
		nelle slides è un'ottima idea, senza appellarsi troppo ai siti, visto che
		ogni tanto ci capita di essere distratti e dimenticarci di aggiornare la
		lista. E con questo intendo che potrebbe sì esserci roba in meno nella lista,
		ma anche contenuti in più, che magari quest'anno abbiamo deciso di non fare.
		
		Immagino
		l'avrete sentito ogni volta che inizia un esame, però un'idea intelligente
		è quella di fare esercizi ogni volta, guardare (neanche studiare, dateci anche
		solo un'occhiata) ai vari materiali presentati in classe ogni settimana e
		vedete se capite tutto. Se avete problemi fate domande in classe, oppure
		mandatemi una mail per vedere quando è possibile vederci a ricevimento.
		
		Tra l'altro non ho deciso un momento specifico per fare ricevimento perché
		non so se posso essere sempre disponibile in un dato momento della settimana,
		per non parlare del fatto che potrebbe essere una palla per voi, visto che
		potrebbe andare a cozzare con altre lezioni.

\chapter{Let's rock!}
adsf

\begin{Verbatim}[label=test]
\renewcommand\chapterheadstart{\vspace*{\beforechapskip}}
\renewcommand\printchaptername{\chapnamefont \@chapapp}
\renewcommand\chapternamenum{\space}
\renewcommand\printchapternum{\chapnumfont \thechapter}
\renewcommand\afterchapternum{\par\nobreak\vskip \midchapskip}
\renewcommand\printchapternonum{}
\renewcommand\printchaptertitle[1]{\chaptitlefont #1}
\renewcommand\afterchaptertitle{\par\nobreak\vskip \afterchapskip}
\end{Verbatim}

\end{document}
