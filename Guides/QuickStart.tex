\documentclass[a4paper]{memoir}

\usepackage[colorlinks]{hyperref}

\title{Architettura degli Elaboratori \\ Quick-Start Guide \\ A.A. 2012/2013}
\author{Michele ``Jazzinghen'' Bianchi \\ eMail: \url{jazzinghen@gmail.com} \\%
	Website \url{http://disi.unitn.it/~bianchi/}}

\usepackage{color,calc,graphicx,soul,fourier}
\usepackage[utf8]{inputenc}
\usepackage[T1]{fontenc}
\usepackage[italian]{babel}

\usepackage{graphicx}

% args: image path
% caption
% width
% label
\newcommand{\image}[4]{%
    \begin{figure}[hbt]%
    \centering%
    \includegraphics[width=#3\columnwidth]{#1}%
    \caption{\emph{#2}}%
    \label{#4}%
    \end{figure}%
}

% args: image path
% caption
% label
\newcommand{\imageFullWidth}[3]{\image{#1}{#2}{0.8}{#3}}
\newcommand{\keyword}[1]{{\color{myorange}#1}}


\usepackage{fancyvrb}
\usepackage[svgnames,dvipsnames]{xcolor}
\definecolor{felinesrcbgcolor}{rgb}{1,1,0.85}
\definecolor{felinesrcbgcolor}{rgb}{0.94,0.97,1}
\definecolor{felineframe}{rgb}{0.79,0.88,1}
\definecolor{myorange}{rgb}{1,0.375,0}
\fvset{frame=lines,
  framesep=3mm,
  framerule=3pt,
  fontsize=\small,
  rulecolor=\color{myorange},
  formatcom=\color{DarkGreen},
}
\lstset{language=[ANSI]C, classoffset=0,
  morekeywords={uint32_t}, keywordstyle=\color{myorange}\textbf}

\hypersetup{
	urlcolor=myorange
}

\definecolor{nicered}{rgb}{.647,.129,.149}
\makeatletter
\newlength\dlf@normtxtw
\setlength\dlf@normtxtw{\textwidth}
\def\myhelvetfont{\def\sfdefault{mdput}}
\newsavebox{\feline@chapter}
\newcommand\feline@chapter@marker[1][4cm]{%
	\sbox\feline@chapter{%
	\resizebox{!}{#1}{\fboxsep=1pt%
	\colorbox{nicered}{\color{white}\bfseries\sffamily\thechapter}%
	}}%
	\rotatebox{90}{%
	\resizebox{%
	\heightof{\usebox{\feline@chapter}}+\depthof{\usebox{\feline@chapter}}}%
	{!}{\scshape\so\@chapapp}}\quad%
	\raisebox{\depthof{\usebox{\feline@chapter}}}{\usebox{\feline@chapter}}%
}
\newcommand\feline@chm[1][4cm]{%
\sbox\feline@chapter{\feline@chapter@marker[#1]}%
\makebox[0pt][l]{% aka \rlap
\makebox[1cm][r]{\usebox\feline@chapter}%
}}
\makechapterstyle{daleif1}{
	\renewcommand\chapnamefont{\normalfont\Large\scshape\raggedleft\so}
	\renewcommand\chaptitlefont{\normalfont\huge\bfseries\scshape\color{nicered}}
	\renewcommand\chapternamenum{}
	\renewcommand\printchaptername{}
	\renewcommand\printchapternum{\null\hfill\feline@chm[2.5cm]\par}
	\renewcommand\afterchapternum{\par\vskip\midchapskip}
	\renewcommand\printchaptertitle[1]{\chaptitlefont\raggedleft ##1\par}
}
\makeatother
\chapterstyle{daleif1}

\begin{document}

\maketitle

\chapter*{Prima d'iniziare}
	
	
	
	\section*{Informazioni sul corso}
	
		Partiamo dalla cosa più importante: come passare l'esame.
		
		Il corso prevede ``solo'' una prova scritta che conterrà un po' tutto quello
		che vedrete (avete visto) nel corso, dalla teoria agli esercizi. Per avere
		una lista di quello che potrebbe esserci controllate sul sito di Luigi
		\footnote{\url{http://disi.unitn.it/~palopoli/courses/Arch/syllabus.html}}
		anche se, come regola di base, direi che sapere le cose che sono presenti
		nelle slides è un'ottima idea, senza appellarsi troppo ai siti, visto che
		ogni tanto ci capita di essere distratti e dimenticarci di aggiornare la
		lista. E con questo intendo che potrebbe sì esserci roba in meno nella lista,
		ma anche contenuti in più, che magari quest'anno abbiamo deciso di non fare.
		
		Immagino
		l'avrete sentito ogni volta che inizia un esame, però un'idea intelligente
		è quella di fare esercizi ogni volta, guardare (neanche studiare, dateci anche
		solo un'occhiata) ai vari materiali presentati in classe ogni settimana e
		vedete se capite tutto. Se avete problemi fate domande in classe, oppure
		mandatemi una mail per vedere quando è possibile vederci a ricevimento.
		
		Tra l'altro non ho deciso un momento specifico per fare ricevimento perché
		non so se posso essere sempre disponibile in un dato momento della settimana,
		per non parlare del fatto che potrebbe essere una palla per voi, visto che
		potrebbe andare a cozzare con altre lezioni.
		
	\section*{Oh, aspetta un secondo!}
	
		Quello che è scritto in questi documenti è tutto frutto della mia mente,
		prodotto in momenti di alti e bassi caffeinici, intervallati dalla scrittura
		di codice a basso livello per simulazioni fisiche o studi sull'interazione
		uomo-macchina.
		
		A causa di questa instabilità potrei aver scritto minchiate, fatto errori
		grammaticali, ecc. Ma la cosa più
		importante è che le opinioni espresse nei miei documenti non riflettono
		necessariamente quelle del professore o dell'università.

\chapter{Let's rock!}

	Siccome il corso prevede un bel po' di ore di studio ed esercizi su Assembly x86\_{}64
	(l'assembly per processori a 64bit), abbiamo deciso di fornirvi di una virtual
	machine per fare tutti i test che volete senza spaccarvi troppo la testa a settare
	i vostri sistemi operativi per compilare software in assembly.
	
	Ora, questa cosa sarebbe una cazzata su delle macchine che fanno girare Linux, visto
	che basta installare \texttt{gcc}, \texttt{gdb} ed altre cosucce ed è fatta, del tipo...
	
	\begin{Verbatim}[label=È tutto più semplice con linux quando si parla di programmazione]
$ sudo apt-get install build-essential gdb
	\end{Verbatim}
	%$
	
	...solo che su altri sistemi (i.e. Windows, MacOS) la cosa non è così semplice\footnote{Dannato
	MacOSX, quante bestemmie mi fai tirare ogni volta che ti accendo.}.
	
	Installare VirtualBox\footnote{\url{https://www.virtualbox.org/}} e far partire una macchina
	virtuale con Ubuntu Linux 12.10 Server edition
	a 64bit, invece, è più semplice e vi da molta più libertà di fare casini, visto che, al massimo,
	vi basta riscaricare l'immagine dal mio sito e ripartire da zero.
	
	Ed ora arriva il bello (brutto?): a causa delle limitate potenze delle macchine virtuali
	(o, se volete, a causa dell'enorme potenza buttata per simulare un sistema che la vostra macchina
	non è, anche se è un sistema a 64bit fatto girare in un sistema a 64bit) il sistema
	operativo che vi sto fornendo NON ha un'interfaccia grafica, quindi faremo come se fosse il 1990.
	
	Oddio, con zshell estesa, gcc 4.7, vim con syntax highlight, elinks, supporto per il mouse ed altre cose
	che ora non ricordo, però senza interfaccina figa, tipo \ref{img:QSCoolInt}. Magari sarà un casino
	all'inizio, ma vi assicuro che ci prenderete la mano. Inoltre avere un sistema virtuale che parte
	in meno di dieci secondi non è male.
	
	\image{Images/QS_CoolInterfaceExample.png}{Uno screenshot del mio sistema operativo: %
	parlo tanto ma, alla fine, non farei mai a meno dell'interfaccia grafica. Non siamo mica ai tempi del %
	DOS, cazzo.}{0.8}{img:QSCoolInt}
	
	E, detto questo, partiamo.
	
	\section{Installate VirtualBox e scaricate l'immagine del sistema dal mio sito}
	
		Fatto? Bene.
		
	\section{Aggiungete la macchina alla lista e preparatela per l'utilizzo}
		
		OK, minchiate a parte, passiamo ad aggiungere la virtual machine e farla partire. Aprite il programma
		ed andate sotto \texttt{File->Import Appliance...}, selezionate \texttt{Open Appliance...}.
		
		Selezionate il file \texttt{.ova} scaricato dal sito ed andate avanti. Dovreste raggiungere
		una schermata simile a questa \ref{img:VBfirst}. Lasciate così com'è e premete su \texttt{Import}.
		
		\image{Images/VBfirst.png}{Per sapere perché la macchina virtuale si chima \emph{OZ-06MS Leo}, guardate %
		su \url{http://tvtropes.org/pmwiki/pmwiki.php/Main/MadeOfExplodium} sotto anime oppure su %
		\url{http://gundam.wikia.com/wiki/OZ-06MS_Leo}. Non prendetelo troppo sul serio (Also: LOL, GUNDAM).}{0.8}{img:VBfirst} 
		
		Fatto questo avrete una macchina virtuale nella lista. Ora potete aggiungere delle cartelle condivise
		tra la macchina virtuale e la vostra macchina. Per fare questo andate su \texttt{Settings}, quindi
		\texttt{Shared Folders}. Se arrivano messaggi sull'accesso delle USB... Lasciate stare, le Shared
		Folders sono più comode.
		
		Comunque, una volta quì aggiungete quante cartelle condivise volete. Vi spiegherò poi come accedervi.
		
	\section{Primi passi in giro per il sistema}
	
		Facciamo partire 'sta macchina e ``Let's have a grand ol' time''! :D
	
		Dovreste arrivare ad una schermata tipo questa entro poco tempo...
		
		\begin{Verbatim}[label=Login + Password]
Ubuntu 12.10 OZ-06MS Leo tty1
OZ-06MS Leo Login: cortana
Password: motherfuckingspace
		\end{Verbatim}

		Non scherzo, quelli sono veramente username e password\footnote{Tra l'altro il sistema non vi mostrerà
		nulla durante l'inserimento della password per... Sicurezza, immagino.}.
		
		Se avete fatto tutto correttamente dovreste trovarvi nel sistema:

		\begin{Verbatim}[label=Party like it's 1999!]
Welcome back, time to do some bloody
 ____   ____ ___ _____ _   _  ____ _____ _ 
/ ___| / ___|_ _| ____| \ | |/ ___| ____| |
\___ \| |    | ||  _| |  \| | |   |  _| | |
 ___) | |___ | || |___| |\  | |___| |___|_|
|____/ \____|___|_____|_| \_|\____|_____(_)

cortana at OZ-06MS in ~
$ 
		\end{Verbatim}
%$
		
		``ZOMMIODDIO, LAINUX!''\footnote{Cit. da me, la prima volta che ho visto una cosa simile. Era una debian
		nel 2003, mi pare.}. Per vedere su che linux siete:
		
		\begin{Verbatim}
cortana at OZ-06MS in ~
$ uname -a

cortana at OZ-06MS in ~
$ Linux OZ-06MSLeo 3.5.0-22-generic #34-Ubuntu SMP Tue Jan 8 21:47:00
  UTC 2013 x86_64 x86_64 x86_64 GNU/Linux
		\end{Verbatim}
%$

		Bien. Se siamo arrivati a questo punto vuol dire che tutto va correttamente. Ora potete
		rozzare in giro per il sistema ed abituarvi a lavorare su riga di comando\footnote{In caso vogliate un
		crash course su come si lavora via riga di comando vi consiglierei di dare un occhiata a risorse online
		tipo \url{http://cli.learncodethehardway.org/}. Ricordatevi sempre che Google-sensei è sempre lì pronto
		a darvi una mano.}
		
	\section{Ok, ma io devo farci degli esercizi su questa macchina.}
		
		Esatto!
		
		È per quello che è stato installato tutto ciò che è necessario per scrivere codice in C ed in Assembly,
        compilarlo e debuggarlo. Per iniziare a scrivere del codice aprite vim...
        
        \begin{Verbatim}[label={LOL, VIM}]
cortana at OZ-06MS in ~
$ vim test.c
		\end{Verbatim}
%$
		
		Ed ecco, sento già arrivare la flame war. Se qualcuno si chiedesse perché non ho installato \texttt{Emacs}
		sulla macchina (\ref{img:YUNO}), provi a lanciare \texttt{df -h} e poi \texttt{sudo apt-get install emacs24-nox},
		come dimostrato sotto:
		
		\begin{Verbatim}[label={Ecco, vedi che Michele è il solito estremista di VIM?}]
cortana at OZ-06MS in ~
$ df -h
Filesystem      Size  Used Avail Use% Mounted on
/dev/sda2       1.8G  1.4G  269M  85% /
udev            237M  4.0K  237M   1% /dev
tmpfs           128M     0  128M   0% /tmp
tmpfs            99M  308K   98M   1% /run
none            5.0M     0  5.0M   0% /run/lock
none            246M     0  246M   0% /run/shm
none            100M     0  100M   0% /run/user
VB              283G   71G  213G  25% /media/sf_VB

cortana at OZ-06MS in ~
$ sudo apt-get install emacs24-nox
Reading package lists...
Building dependency tree...
Reading state information...
The following extra packages will be installed:
  emacs24-bin-common emacs24-common emacs24-common-non-dfsg emacsen-common
  libasound2 libxml2 sgml-base xml-core
Suggested packages:
  emacs24-el libasound2-plugins libasound2-python sgml-base-doc debhelper
The following NEW packages will be installed
  emacs24-bin-common emacs24-common emacs24-common-non-dfsg emacs24-nox
  emacsen-common libasound2 libxml2 sgml-base xml-core
0 upgraded, 9 newly installed, 0 to remove and 5 not upgraded.
Need to get 28.9 MB of archives.
After this operation, 87.4 MB of additional disk space will be used.
Do you want to continue [Y/n]? n
		\end{Verbatim}
%$
		
		\image{Images/YUNOEmacs.jpg}{Memegenerator. Sigh. E pubblicizzano... IL NUOVO \href{http://www.4chan.org/}{4chan}! %
			Ma che diavolo?}{0.5}{img:YUNO}
        
        Ok, scherzi a parte. Non c'è moltissimo spazio sulla macchina virtuale, perché ho provato a renderla
        il più piccola/leggera possibile, fallendo miseramente e creando una macchina da 700MB. Se volete
        installarlo potete, visto che occupa ``solo'' 88MB. Inoltre, per chi volesse c'è anche \texttt{nano}.
        
        A difesa di \texttt{VIM} devo dire che può fare un po' più di cose, tipo tabbed editing, splitting
        ricorsivo delle finestre, chiamate dirette a comandi della shell, ecc. ecc.
        
        Non starò qui a spiegarvi tutto sull'editor perché ci vorrebbe l'intero corso, perciò il consiglio è sempre
        quello di chiedere a Google-sensei\footnote{Partite da \url{http://www.swaroopch.com/notes/Vim_en-Table_of_Contents/}}
        e che, per scrivere, si preme \texttt{i}, per salvare si fa \texttt{[ESC] -> :w}, per uscire \texttt{[ESC] -> :q} 
        e per salvare ed uscire \texttt{[ESC] -> :x}.
        
	\section{Ma torniamo a fare la prima prova}
		
		Yeah. Allora, aperto \texttt{VIM} dovremmo già avere del testo, tipo:
		
	    \begin{Verbatim}[label={UNF, DAT syntax highlighting (dentro VIM)}]
#include <stdlib.h>
#include <stdio.h>    // I/O lib
#include <stdint.h>   // Standard variable types

int main (int argc, char **argv){
    printf("Well, that was easy.\n");
    return 0;
}
		\end{Verbatim}
		
		Ecco, come avrete già visto, questo è lo scheletro di un programma in C. Ci sono le librerie base incluse,
		c'è il main, ecc. ecc.

\end{document}