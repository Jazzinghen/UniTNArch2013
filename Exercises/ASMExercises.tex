\documentclass[a4paper]{memoir}

\usepackage[colorlinks]{hyperref}

\title{Architettura degli Elaboratori \\ Assembly Exercises \\ A.A. 2012/2013}
\author{Michele ``Jazzinghen'' Bianchi \\ eMail: \url{jazzinghen@gmail.com} \\%
	Website \url{http://disi.unitn.it/~bianchi/}}

\usepackage{color,calc,graphicx,soul,fourier}
\usepackage[utf8]{inputenc}
\usepackage[T1]{fontenc}
\usepackage[italian]{babel}
\usepackage{listings}



\usepackage{graphicx}

% args: image path
% caption
% width
% label
\newcommand{\image}[4]{%
    \begin{figure}[hbt]%
    \centering%
    \includegraphics[width=#3\columnwidth]{#1}%
    \caption{\emph{#2}}%
    \label{#4}%
    \end{figure}%
}

% args: image path
% caption
% label
\newcommand{\imageFullWidth}[3]{\image{#1}{#2}{0.8}{#3}}
\newcommand{\keyword}[1]{{\color{myorange}#1}}


\usepackage{fancyvrb}
\usepackage[svgnames,dvipsnames]{xcolor}
\definecolor{felinesrcbgcolor}{rgb}{1,1,0.85}
\definecolor{felinesrcbgcolor}{rgb}{0.94,0.97,1}
\definecolor{felineframe}{rgb}{0.79,0.88,1}
\definecolor{myorange}{rgb}{1,0.375,0}
\fvset{frame=lines,
  framesep=3mm,
  framerule=3pt,
  fontsize=\small,
  rulecolor=\color{myorange},
  formatcom=\color{DarkGreen},
}
\lstset{language=[ANSI]C, classoffset=0,
  morekeywords={uint32_t}, keywordstyle=\color{myorange}\textbf}

\hypersetup{
	urlcolor=myorange
}

\definecolor{nicered}{rgb}{.647,.129,.149}
\makeatletter
\newlength\dlf@normtxtw
\setlength\dlf@normtxtw{\textwidth}
\def\myhelvetfont{\def\sfdefault{mdput}}
\newsavebox{\feline@chapter}
\newcommand\feline@chapter@marker[1][4cm]{%
	\sbox\feline@chapter{%
	\resizebox{!}{#1}{\fboxsep=1pt%
	\colorbox{nicered}{\color{white}\bfseries\sffamily\thechapter}%
	}}%
	\rotatebox{90}{%
	\resizebox{%
	\heightof{\usebox{\feline@chapter}}+\depthof{\usebox{\feline@chapter}}}%
	{!}{\scshape\so\@chapapp}}\quad%
	\raisebox{\depthof{\usebox{\feline@chapter}}}{\usebox{\feline@chapter}}%
}
\newcommand\feline@chm[1][4cm]{%
\sbox\feline@chapter{\feline@chapter@marker[#1]}%
\makebox[0pt][l]{% aka \rlap
\makebox[1cm][r]{\usebox\feline@chapter}%
}}
\makechapterstyle{daleif1}{
	\renewcommand\chapnamefont{\normalfont\Large\scshape\raggedleft\so}
	\renewcommand\chaptitlefont{\normalfont\huge\bfseries\scshape\color{nicered}}
	\renewcommand\chapternamenum{}
	\renewcommand\printchaptername{}
	\renewcommand\printchapternum{\null\hfill\feline@chm[2.5cm]\par}
	\renewcommand\afterchapternum{\par\vskip\midchapskip}
	\renewcommand\printchaptertitle[1]{\chaptitlefont\raggedleft ##1\par}
}
\makeatother
\chapterstyle{daleif1}

\begin{document}

\maketitle

\chapter*{Introduzione}

  Ho messo in questo documento tutti gli esercizi di assembly usati negli anni passati che sono riuscito
  a recuperare anche grazie all'aiuto di Luigi Palopoli e Federico Moro.
  
  Ho fatto anche un capitolo che contiene dei suggerimenti per la scrittura degli esercizi (tutti i suggerimenti
  che mi sono venuti in mente, in realtà) ed uno con UNA delle possibili soluzioni per esercizio.
  
  Ho convertito alcune parti degli esercizi, usando tipi standardizzati (e.g. {\color{myorange}\texttt{int32\_t}}
  invece che {\color{myorange}\texttt{int}}). Non cambia
  molto, rende solo il tutto più formale. Scusate, pippe da scrittore di codice embedded XD
  
  Come al solito, se avete domande sugli esercizi mandatemi due righe, che vediamo di risolvere.
  
\chapter{Esercizi degli anni passati}
	
	Come spiegato in classe io vi consiglierei di seguire questi tre passaggi per risolvere gli esercizi di scrittura
	di programmi in Assembly:
	
	\begin{itemize}
    \item Scrivere il programma in C, così da avere bene in mente le variabili necessarie, di che tipo sono e così via
    \item Trasformare il programma in una versione che usa solo GOTO\footnote{Che Sebastiani abbia pietà di me}, così da sapere
      quali e quante labels utilizzare e che controlli fare
    \item Trasformare il codice da C ad Assembly
	\end{itemize}
	
	Ovviamente non vi è richiesto all'esame di fare tutto questo, potete anche scrivere il codice direttamente in Assembly,
	però vi assicuro che rischiate errori.
	
	Guardate me. In classe faccio errori anche seguendo tutti i passaggi.
	
	\section{Giugno 2011}
    Ah, un classico:
    
    Scrivere in codice assembly la funzione
    
    \begin{lstlisting}
uint32_t fib (uint32_t n)
    \end{lstlisting}
    
    Che restituisce l'\keyword{n}esimo numero della serie di Fibonacci.
    
    La funzione dev'essere scritta in forma ricorsiva. Visualizzate inoltre la situazione dello stack (evidenziando i
    differenti stack frames) nel momento di massima espansione della chiamata \keyword{fib(3)}.
    
    \subsection{Bonus}
    
      Scrivete la versione iterativa e spiegate in cosa è migliore rispetto a quella ricorsiva.
      
	\section{Luglio 2011}
	
	  Scrivere in codice assembly la funzione
    
    \begin{lstlisting}
uint32_t is_prime (uint32_t n)
    \end{lstlisting}
    
    Che restituisce 1 se il \keyword{n} è primo, 0 altrimenti.
    
    \subsection{Bonus}
    
      Scrivete la funzione in forma ricorsiva. Potete aggiungere un parametro alla funzione.
      
  \section{Settembre 2011}
    Scrivere in codice assembly la funzione
    
    \begin{lstlisting}
uint32_t count_even (const int32_t * a, uint32_t l)
    \end{lstlisting}
    
    Che restituisce il conteggio dei numeri primi contenuti nell'array \keyword{a}, di lunghezza \keyword{l}.\footnote{\keyword{a}
    è dichiarato come \keyword{const uint32\_t}. Questo significa solo che la funzione NON va a modificare i contenuti dell'array}
    
    \subsection{Bonus}
    
      Utilizzate una sola operazione per capire se il numero è pari.
      
	\section{Giugno 2012}
	
	  Scrivere in codice assembly la funzione
    
    \begin{lstlisting}
uint32_t pal_check (const char * s)
    \end{lstlisting}
	
	  Che restituisce 1 se la stringa \keyword{s} è palindoma\footnote{Ovvero è uguale leggerla da destra a sinistra o da
	  sinistra a destra (e.g. Otto, Radar, Nisioisin)}, 0 altrimenti.
	  
	\section{Luglio 2012}
	
	  Scrivere in codice assembly la funzione
    
    \begin{lstlisting}
uint32_t is_suffix (const char * s, const char * t)
    \end{lstlisting}
    
    Che restituisce 1 se la stringa \keyword{t} è un suffisso della stringa \keyword{s}, 0 altrimenti.
    
    Assumete, per semplicità, che \keyword{s} è sempre più lungo di \keyword{t} e che entrambe le stringhe contengano
    almeno un carattere.
    
    \subsection{Bonus}
      
      Non assumete un bel nulla sulle stringhe passate alla funzione.
      
	\section{Settembre 2012}
	
	  Scrivere in codice assembly la funzione
    
    \begin{lstlisting}
void invert (const int32_t * s, int32_t * d, uint32_t c)
    \end{lstlisting}
    
    Che copi l'array \keyword{s} di lunghezza \keyword{c} in ordine invertito in \keyword{d}.
    
    Assumete (anche perché non possiamo fare altrimenti, LOL) che l'array \keyword{d} sia di lunghezza corretta.
    
    \subsection{Bonus}
    
      Ok, diciamo che abbiamo scritto una libreria che utilizza un formato per gli array numerici in modo
      che siano simili alle stringhe, ovvero che sono terminati da un numero specifico (sceglietelo voi: può essere 42, 1337,
      214134). Riscrivete la funzione:
      
      \begin{lstlisting}
uint32_t invert (const int32_t * s, int32_t * d, uint32_t c)
      \end{lstlisting}
      
      Tale che restituisca 1 se riusciamo a fare la copia, 0 altrimenti (i.e. se usciamo dall'array prima di aver finito
      di invertire l'array.
      
	\section{Gennaio 2013}
	
	  Scrivere in codice assembly la funzione
    
    \begin{lstlisting}
uint32_t str_cmp (const char * a, const char * b)
    \end{lstlisting}
    
    Che confronti le stringhe \keyword{a} e \keyword{b} e che restituisca il numero di caratteri uguali nella stessa posizione.
    Hint: '\\0' non è proprio un carattere simile :D    
    
    Ad esempio:
    
    \begin{tabular}{|c|c|c|c|}
    \hline 
    \keyword{a} & \texttt{abcde} & \texttt{safvbe} & \texttt{oihfcapinwq} \\ 
    \keyword{b} & \texttt{abbfe} & \texttt{tafede} & \texttt{maefpasidwc} \\ 
    \hline 
    risultato & 2 & 3 & 4 \\ 
    \hline 
    \end{tabular}
    
    Assumete che le stringhe siano lunghe uguali.
    
    \subsection{Bonus}
    
      Al solito non assumete che siano uguali. Restituite il conteggio dei caratteri uguali finchè potete.
      OPPURE!
      
      Scrivere in codice assembly la funzione\footnote{Spoilers: Copia Incolla}
    
    \begin{lstlisting}
int32_t str_cmp (const char * a, const char * b)
    \end{lstlisting}
    
      Che si comporti come prima, ma restituisca -1 se incontra qualche problema.
      
	\section{Febbraio 2013}
	
	  Scrivere in codice assembly la funzione
    
    \begin{lstlisting}
int32_t bounded_sum (const int32_t * a, uint32_t l, int32_t t)
    \end{lstlisting}
    
    Che restituisca la somma di tutti i numeri dispari minori di \keyword{t} contenuti nell'array \keyword{a} di lunghezza
    \keyword{l}.
    
%\chapter{Hints}

	%Arriveranno, non vi preoccupate.

%\chapter{Soluzioni}

  %Anche queste, solo che devo scrivermele giù, sorry.
\end{document}