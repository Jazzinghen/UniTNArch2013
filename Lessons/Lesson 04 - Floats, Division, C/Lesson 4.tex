\documentclass{beamer}
\usetheme[pageofpages=of,% String used between the current page and the
                         % total page count.
          alternativetitlepage=true,% Use the fancy title page.
          ]{Torino}

% Nouvelle is a green and red alternative to the chameleon color theme.
\usecolortheme{nouvelle}
\geometry{paperwidth=140mm,paperheight=105mm}
\usepackage[utf8]{inputenc}
\usepackage[T1]{fontenc}
\usepackage[italian]{babel}
\usepackage{soul}
\usepackage{graphicx}
\usepackage{algorithm2e}
\usepackage{multirow}

\usepackage{fancyvrb}
\definecolor{felinesrcbgcolor}{rgb}{1,1,0.85}
\definecolor{felinesrcbgcolor}{rgb}{0.94,0.97,1}
\definecolor{felineframe}{rgb}{0.79,0.88,1}
\definecolor{myorange}{rgb}{1,0.375,0}
\fvset{frame=lines,
  framesep=3mm,
  framerule=3pt,
  fontsize=\small,
  rulecolor=\color{myorange},
  formatcom=\color{DarkGreen},
}

\hypersetup{
	urlcolor=myorange
}

\title[Arch2013] % (optional, only for long titles)
{Architettura degli elaboratori 2012/2013}
\subtitle{Aritmetica binaria: Divisione Float e Float vs. C/C++}
\author{Michele ``Jazzinghen'' Bianchi\inst{1}}
\institute[DISI] % (optional)
{
  \inst{1}%
  Dipartimento di Ingegneria e Scienze dell'Informazione\\
  Universtià degli Studi di Trento
}
\date[2013-03-20] % (optional)
{20 Marzo 2013}
\subject{Computer Science, Embedded Systems}

\begin{document}
	\frame{\titlepage}
	\section{It's answer TIME!}
	\begin{frame}
    \frametitle{Alcune cose prima d'iniziare}
		\begin{itemize}
			\item La sottrazione può venir fatta sommando il complemento a
				2 solo se il numero di bit è limitato.
		  \item Quando facciamo operazioni con i float aggiungiamo zeri
		  		in fondo in modo da riempire tutti gli spazi della mantissa
		\end{itemize}
	\end{frame}   
  
	\section{Riassunto dell'ultima puntata}  
  \begin{frame}
    \frametitle{Cos'è successo l'ultima volta?}
		Abbiamo visto:    
    \begin{itemize}
    		\item Arrotondamento
    		\item Moltiplicazione
    		\item Somma
    		\item Sottrazione
    \end{itemize}
    %Unari, Babilonesi, Romani, Posizioniali (indiani), Binaria
  \end{frame}
	
	\section{Float Operations}
	\subsection{Divisione}
  \begin{frame}
    \frametitle{Floating Point}
    \framesubtitle{Divisione}       
    
    $$\frac{(-1)^{s1} \text{ } M1 \text{ } 2^{Exp1}}{(-1)^{s2} \text{ } M2 \text{ } 2^{Exp2}}= (-1)^{s} \text{ } M \text{ } 2^{Exp}$$
  	
  		\begin{center}
  			\begin{tabular}{|cc|}
  			\hline 
  			Segno (s) & s1 $\oplus$ s2 \\ 
  			\hline 
  			Valore (M) & $\frac{\text{M1}}{\text{M2}}$ \\ 
  			\hline 
  			Esponente (Exp) & Exp1 $-$ Exp2 \\ 
  			\hline 
  			\end{tabular} 
  		\end{center}
	  
	  \vspace{1em}
	  
	  Normalizzazione:
	  \begin{itemize}
	  		\item Se il numero è $> 1_{2}$ o $\geq 2_{10}$, shiftate a destra ed incrementate Exp di uno.
	  		\item Se l'esponente va oltre il massimo andiamo in \emph{overflow}
	  		\item Arrotondiamo il numero in modo che sia contenuto nella mantissa
	  \end{itemize}
  \end{frame}	  
  
  \begin{frame}
    \frametitle{Floating Point}
    \framesubtitle{Divisione tra interi}
    La parte peggiore della divisione tra float è dividere le due mantisse.
		
		\vspace{1em}
		    
    Vediamo prima come funziona la divisione tra interi:

		\vspace{1em}
		
		È uguale alla \emph{Divisione lunga} insegnata nelle scuole, la nostra
		fortuna, però, è che possiamo scegliere solo tra 1 e 0, quidi il divisore
		è o non è contenuto nelle cifre del dividendo che consideriamo ad ogni passaggio.
  \end{frame}
  
  	\subsection{Esempio di divisione base 10}
  \begin{frame}
	    \frametitle{Floating Point}
	    \framesubtitle{Esempio di divisione base 10}
				    
	    Iniziamo col fare una divisione banale in base 10.
	    
	    $$33_{10} / 6_{10} = ?$$
	    
	    \pause
	    \vspace{2em}
	    \begin{center}
	    		\setlength{\tabcolsep}{2pt}
	    		\begin{tabular}{cccccccc}
	    			3 & 3 & : & 6 & = & 5. & 5 \\
	    			3 & 0 &   &   &   &    &   \\
	    			\hline
	    			  & 3 & 0 &   &   &    &   \\
	    			  & 3 & 0 &   &   &    &   \\
	    			\hline
	    			  &   & - &   &   &    &   \\
	    		\end{tabular}
	    \end{center}
	    \pause
	    \vspace{2em}
	    $$33_{10} / 6_{10} = 5.5_{10}$$
	\end{frame}

	\subsection{Esempio di divisione base 2}
  \begin{frame}
	    \frametitle{Floating Point}
	    \framesubtitle{Esempio di divisione base 2}
				    
	    La divisione in base 2 funziona nella stessa maniera.
	    
	    $$100001_{2} / 110_{2} = ?$$
	    
	    \pause
	    \vspace{2em}
	    \begin{center}
	    		\setlength{\tabcolsep}{2pt}
	    		\begin{tabular}{ccc|cccccccc}
	    				&   &   &   &   &   & 1 & 0 & 1. & 1	\\
	    		  \hline    			
	    			1 & 1 & 0 & 1 & 0 & 0 & 0 & 0 & 1. & 0 \\
	    			  &   &   &   & 1 & 1 & 0 &   &   &    \\
	    			\hline
	    			  &   &   &   &   & 1 & 0 & 0 & 1 &    \\
	    			  &   &   &   &   &   & 1 & 1 & 0 &    \\
	    			\hline
	    			  &   &   &   &   &   &   & 1 & 1 & 0  \\
	    			  &   &   &   &   &   &   & 1 & 1 & 0  \\
	    			\hline
	    			  &   &   &   &   &   &   &   &   & -  \\
	    		\end{tabular}
	    \end{center}
	    \pause
	    \vspace{2em}
	    $$100001_{2} / 110_{2} = 101.1_{2}$$
	\end{frame}


	\subsection{Esempio di divisione float}
  \begin{frame}
    \frametitle{Floating Point}
    \framesubtitle{Esempio di divisione float}
    
		$$330_{10} / -0.625_{10} = ?$$
		
		\pause
		
		$330_{10} = 01000011101001010000000000000000$
		
		$-0.625_{10} = 10111111001000000000000000000000$
		
		\begin{center}
    		\setlength{\tabcolsep}{2pt}
    		\begin{tabular}{ccc|cccccccc}
    				&   &   & 1.& 0 & 0 & 0 & 0 & 1 & 0 & 0 \\
    		  \hline    			
    			1.& 0 & 1 & 1.& 0 & 1 & 0 & 0 & 1 & 0 & 1 \\
    			  &   &   & 1 & 0 & 1 &   &   &   &   &   \\
    			\hline
    			  &   &   &   &   & 0 & 0 & 0 & 1 & 0 & 1 \\
    			  &   &   &   &   &   &   &   & 1 & 0 & 1 \\
    			\hline
    			  &   &   &   &   &   &   &   &   &   & - \\
    		\end{tabular}
	  \end{center}		
		
		$$\text{M} = 00001000000000000000000$$ 
    
  \end{frame} 
  
  \begin{frame}
  	  \frametitle{Floating Point}
    \framesubtitle{Esempio di divisione float, cont.}
  	
    $$\text{Sign} = 0 \oplus 1 = 1$$
    
    $$\text{Exp} = 135 - 126 + 127 = 136_{10} = 10001000_{2}$$
    
    \pause
    \vspace{2em}
    
    $$11000100000001000000000000000000 = -528_{10}$$
  		
  \end{frame}
  \section{I Float e C/C++}
  \subsection{Le basi}
  \begin{frame}
  	  \frametitle{Floats vs. C/C++}
    \framesubtitle{I float nel C/C++}
  	
    Presenti due versioni del floating point:
    \begin{itemize}
    		\item \texttt{float}, 32-bit
    		\item \texttt{double}, 64-bit
    \end{itemize}
        
    \vspace{1em}
    
    Il casting tra \texttt{int}, \texttt{float} e \texttt{double} varia la rappresentazione in bit:
    \begin{itemize}
    		\item \texttt{float}\\ \texttt{double} -> \texttt{int} troncamento dei bit
    		\begin{itemize}
    			\item In pratica è un arrotondamento per difetto
    			\item Se è fuori scala o NaN allora viene settato a \texttt{INT\_{}MIN}
    		\end{itemize}
    		\item \texttt{int} -> \texttt{double}: Conversione esatta finché l'\texttt{int} ha $\leq 53$ cifre significative.
    		\item \texttt{int} -> \texttt{float}: Arrotondato in base al metodo scelto (normalmente arrotondamento al più vicino)
    \end{itemize}
    
  \end{frame}
  \subsection{Quiz}
  \begin{frame}
  	  \frametitle{Floats vs. C/C++}
    \framesubtitle{Discutiamo un po'...}
  	
    Proviamo a controllare se le seguenti espressioni sono sempre vere. Discutiamo perché.
        
    \vspace{1em}
    \begin{columns}
			\column{.3\textwidth}
			\begin{center}
				\texttt{int x}
	
			  \texttt{float f}
				
				\texttt{double d}			
			\end{center}						
			\column{.7\textwidth}	    
	    \begin{itemize}
	    		\item \texttt{x == (int)(float) x}
				\item \texttt{x == (int)(double) x}
				\item \texttt{f == (float)(double) f}
				\item \texttt{d == (float) d}
				\item \texttt{f == -(-f)}
				\item \texttt{2/3 == 2/3.0}
				\item \texttt{d < 0.0} -> \texttt{((d*2) < 0.0)}
				\item \texttt{d > f} -> \texttt{-f > -d}
				\item \texttt{d * d >= 0.0}
				\item \texttt{(d+f)-d == f}
	    \end{itemize}
    \end{columns}
  \end{frame}
% etc
\end{document}
