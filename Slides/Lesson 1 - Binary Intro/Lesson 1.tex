\documentclass{beamer}
\usetheme{CambridgeUS}
\usecolortheme{beaver}
\usepackage[utf8]{inputenc}
\usepackage[T1]{fontenc}
\usepackage[italian]{babel}
\usepackage{soul}

\usepackage{fancyvrb}
\definecolor{felinesrcbgcolor}{rgb}{1,1,0.85}
\definecolor{felinesrcbgcolor}{rgb}{0.94,0.97,1}
\definecolor{felineframe}{rgb}{0.79,0.88,1}
\definecolor{myorange}{rgb}{1,0.375,0}
\fvset{frame=lines,
  framesep=3mm,
  framerule=3pt,
  fontsize=\small,
  rulecolor=\color{myorange},
  formatcom=\color{DarkGreen},
}

\hypersetup{
	urlcolor=myorange
}

\title[Arch2013] % (optional, only for long titles)
{Architettura degli elaboratori 2012/2013}
\subtitle{Rappresentazione ed aritmetica binaria di base}
\author{Michele ``Jazzinghen'' Bianchi\inst{1}}
\institute[DISI] % (optional)
{
  \inst{1}%
  Dipartimento di Ingegneria e Scienze dell'Informazione\\
  Universtià degli Studi di Trento
}
\date[FEB 2013] % (optional)
{21 Febbraio 2013}
\subject{Computer Science, Embedded Systems}

\begin{document}
	\frame{\titlepage}
	\section{Intro}
	\begin{frame}
    \frametitle{Introduzione}
		\begin{itemize}
			\item Contatti: \url{michele.bianchi@unitn.it}
		  \item Sito web con i materiali: \url{http://disi.unitn.it/~bianchi}
		  \item Ricevimento: scrivetemi una mail, così ci mettiamo d'accordo.
		  \item Domande o parti poco chiare: Ovviamente sono qui per questo, quindi
		    chiedete pure, altrimenti mandatemi le domande via mail, alle quali
		    risponderò all'inizio della lezione successiva
		\end{itemize}    
   
  \end{frame}
  \begin{frame}
    \frametitle{Introduzione}
    \framesubtitle{Extra}
    \begin{itemize}
    		\item Sulla mia pagina è presente una Quick Start Guide per quanto riguarda
    			la seconda parte del corso. Vi potrebbe essere utile leggerla subito, così
    			se ci sono casini metto a posto
    		\item Aggiungerò dei link a delle risorse online utili per il corso, dateci
    			un'occhiata ogni tanto. So che vi sentite dire questa cosa spesso, ma \st{a me%
    			servono hit per prendere soldi} nel mio caso vi potrebbe essere utile, visto
    			che non ci sono ancora tutti i contenuti.
    \end{itemize}
  \end{frame}

  \section[AllYourBases...]{Sistemi numerici (Basi)}
	\subsection{Storia della rappresentazione numerica}  
  \begin{frame}
    \frametitle{Storia}
    \begin{itemize}
    		\item Sistemi unari (dita, tacche, ideogrammi)
    		\item Sistemi di numerazione additivi (Romani, Egizi)
    		\item Sistemi letterali (e.g. \emph{Sixty nine}, \emph{Soixante dix-neuf}, \emph{Settantanove})
    		\item Notazioni posizionali (Babilonesi [base60], Hindu-Arabic [base10])
    \end{itemize}
    %Unari, Babilonesi, Romani, Posizioniali (indiani), Binaria
  \end{frame}
	
	\subsection{Basi numeriche informatiche}
  \begin{frame}
    \frametitle{Basi informatiche}
    \begin{itemize}
    		\item Base-2 (Binario): rappresentazione grezza delle linee logiche
    		\item Base-8 (Ottale): rappresentazione di 3 bit (deprecato)
    		\item Base-16 (Esadecimale): rappresentazione 4 bit (e.g. Colori per CSS)
    \end{itemize}
  \end{frame}


	\section[BinaryData]{Rappresentazione Binaria dei dati}
	\subsection{Rappresentazioni di base}  
  \begin{frame}
    \frametitle{Definizioni di base}    	
    \begin{itemize}
    		\item Bit [0,1]: stato di una linea logica
    		\item Byte [0,255]: 8 bit assieme (e.g. 1 carattere ASCII)
    		\item Word [0,$2^{lunghezza}$]: numero variabile di byte a seconda del sistema (e.g. float da 4byte)
    \end{itemize}
  \end{frame}
  \subsection{Standard Types}
  \begin{frame}   	
    \begin{itemize}
    		\item Integer (8, 16, 32, 64 bit)
    			\begin{itemize}
    				\item Unsigned: Da 0 a $2^{bit}-1$
    				\item Signed: Da $-2^{bit-1}$ a $2^{bit-1}-1$
    			\end{itemize}
    		\item Float
    			\begin{itemize}
    				\item Single: $1.175494351e^{-38}$ a $3.402823466e^{38}$
    				\item Double: $2.2250738585072014e^{-308}$ a $1.7976931348623158e^{308}$
    				\item Quad (There is no kill like overkill): $3.362e^{-4932}$ a $1.20e^{4932}$
    			\end{itemize}
    \end{itemize}

		\begin{block}{N.B.}
			Per iniziare lavoreremo con i numeri naturali (interi positivi) a 8 bit. Ovvero da 0 a 255,
			come se stessimo giocando a Final Fantasy VII.
		\end{block}
    %Parlare dei vari tipi, aggiungere i limiti e dire che lavoreremo con gli unsigned.
  \end{frame}
  \subsection{Strutture dati complesse}
  \begin{frame}
    \frametitle{Strutture complesse}
    \begin{itemize}
    		\item Testo: Sequenza di byte o, nel caso di sistemi moderni, di word da 8 a 32 bit (UTF-8)
    		\item Immagini: Bitmap a profondità variabile (1, 8, 16, 24, 32 bit)
    			\begin{itemize}
    				\item 1 bit: Bianco e Nero
    				\item 8 bit: 256 colori (NES! MSX2! :D Glorious Retrogaming)
    				\item 16 bit: 65536 colori (SNES, Windows 95)
    				\item 24 bit: 16 milioni di colori (Circa tutto adesso)
    				\item 32 bit: 16 milioni di colori + Alpha Channel
    			\end{itemize}
    		\item Audio: Dati stereo codificati tramite Pulse Code Modulation (PCM)\footnote{Dannato PCM!}
    \end{itemize}
  \end{frame} 
  
  \section[Conversion]{Conversione di Base}
  \subsection{Algoritmo di conversione}
  \begin{frame}
    \frametitle{Conversione decimale in binario}
    %Content goes here
  \end{frame}
	\begin{frame}
    \frametitle{Conversione binario in decimale}
    %Content goes here
  \end{frame}  
  \subsection{Esempi}
  \begin{frame}
    \frametitle{Esempi}
    %Content goes here
  \end{frame}
  
  \section[Operations]{Aritmetica dei naturali}
	\subsection[NatSum]{Somma di Naturali}  
	  \begin{frame}
	    \frametitle{Somma di naturali}
	    \framesubtitle{Un'introduzione, prima}
	    %Content goes here
	  \end{frame}
	  \begin{frame}
	    \frametitle{Somma di naturali}
	    \framesubtitle{Esempi}
	    %Content goes here
	  \end{frame}
  \subsection[NatSub]{Sottrazione di Naturali}  
	  \begin{frame}
	    \frametitle{Sottrazione di naturali}
	    \framesubtitle{Operazione di base e ``Il prestito''}
	    %Content goes here
	  \end{frame}
	  \begin{frame}
	    \frametitle{Sottrazione di naturali}
	    \framesubtitle{Esempi}
	    %Content goes here
	  \end{frame}
	\subsection[NatMul]{Moltiplicazione di Naturali}  
	  \begin{frame}
	    \frametitle{Moltiplicazione di naturali}
	    \framesubtitle{Moltiplicazione (e divisione) per potenze di due}
	    %Content goes here
	  \end{frame}
	  \begin{frame}
	    \frametitle{Moltiplicazione di naturali}
	    \framesubtitle{Moltiplicazione tra due Naturali}
	    %Content goes here
	  \end{frame}
		\begin{frame}
	    \frametitle{Moltiplicazione di naturali}
	    \framesubtitle{Esempi}
	    %Content goes here
	  \end{frame}
	\subsection[Errors]{Error checking}
		\begin{frame}
	    \frametitle{Error Checking}
	    \framesubtitle{Overflow/Underflow}
	    %Overflow flag/bit
	  \end{frame}
  
  \section[NegativeRep]{Aggiunta del segno negli interi}
  \subsection{Segno-e-Modulo}
  \begin{frame}
    \frametitle{Rappresentazione in Segno-e-Modulo}
    %Content goes here
  \end{frame}
  \subsection{Complemento a 2}
  \begin{frame}
    \frametitle{Complemento a 2}
    %Complemento ad 1 sarebbe figo, ma ci sega un valore. Complemento a 2 (Cmpl1 + 1) rimuove il problema. 
  \end{frame}
% etc
\end{document}